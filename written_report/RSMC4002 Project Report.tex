\documentclass[11pt,a4paper]{article}
\input{myPreliminary}

\newcounter{magicrownumbers}
\newcommand\rownumber{\stepcounter{magicrownumbers}\arabic{magicrownumbers}}


%------------------------------------------------------------------------------



\begin{document}
    % Page 0 for names and table of contents
    \thispagestyle{empty}
    \pagenumbering{gobble} 
    \title{\textsc{RMSC 4002} -- Financial Data Analytics with Machine Learning \\ Group Project}
    \author{
        CHOI Sen Hei (SID: \texttt{1155109412}) \\
        IEONG Hei (SID: \texttt{1155104271}) \\
        LAM Wai Chui (SID: \texttt{1155152095}) \\
        LAU Chiu Tan (SID: \texttt{1155108960}) \\
        LAW Yiu Leung Eric (SID: \texttt{1155149315})
    }
    \date{\today}
    \maketitle
    
    \tableofcontents
    \newpage
    
    
    % Section 1
    \pagenumbering{arabic}
    \setcounter{page}{1}
    
    \section{Principal Component Analysis Factor Models and Recommender Systems}
    \subsection{Pricipal Component Analysis (PCA)}
    
    
    % Section 2
    \newpage
    \section{Classification / Decision and Regression Trees}
    In this section, we are going to use Decision Trees and Random Forest methods to predict binary response variable.
    
    \subsection{Dataset}
    For the dataset, we use \href{https://archive.ics.uci.edu/ml/datasets/bank+marketing}{Bank Marketing Data Set} from  \href{https://archive.ics.uci.edu/ml/index.php}{UCI Machine Learning Repository}.
    The data is related with direct marketing campaigns of a Portuguese banking institution. The marketing campaigns were based on phone calls. Often, more than one contact to the same client was required, in order to access if the product (bank term deposit) would be ('yes') or not ('no') subscribed \cite{MORO201422}. There are total 21 variables (including y, the response variable), 41188 valid records.
    
    \subsection{Feature Description}
    There are 20 explanatory variables, including numerical and categorical variables:
    
    \begin{table}[h]
        \begin{minipage}{.5\linewidth}
            \centering
            \begin{tabular}{r c c}
                 & Feature & Type \\
                \hline \hline
                \rownumber & age & numeric \\
                \rownumber & job & categorical \\
                \rownumber & marital & categorical \\
                \rownumber & education & categorical \\
                \rownumber & default & categorical \\
                \rownumber & housing & categorical \\
                \rownumber & loan & categorical \\
            \end{tabular}
            \caption{Bank client data}\label{tab:bank.client}
        \end{minipage}%
        \begin{minipage}{.5\linewidth}
            \centering
            \begin{tabular}{r c c}
                 & Feature & Type \\
                \hline \hline
                \rownumber & contact & categorical \\
                \rownumber & month & categorical \\
                \rownumber & day\_of\_week & categorical \\
                \rownumber & duration & numerical \\
            \end{tabular}
            \caption{Related with the last contact of the current campaign}\label{tab:last.contact}
        \end{minipage}
    \end{table}
    
    \begin{table}[h]
        \begin{minipage}{.5\linewidth}
            \centering
            \begin{tabular}{r c c}
                 & Feature & Type \\
                \hline \hline
                \rownumber & campaign & numeric \\
                \rownumber & pdays & numerical \\
                \rownumber & previous & numerical \\
                \rownumber & poutcome & categorical \\
            \end{tabular}
            \caption{Other attributes}\label{tab:other}
        \end{minipage}%
        \begin{minipage}{.5\linewidth}
            \centering
            \begin{tabular}{r c c}
                 & Feature & Type \\
                \hline \hline
                \rownumber & emp.var.rate & numerical \\
                \rownumber & cons.price.idx & numerical \\
                \rownumber & cons.conf.idx & numerical \\
                \rownumber & euribor3m & numerical \\
                \rownumber & nr.employed & numerical \\
            \end{tabular}
            \caption{social and economic context attributes}\label{tab:soc.econ}
        \end{minipage}
    \end{table}
    
    \subsection{Project Description}
    We are going to perform an analysis in \textbf{topic 6: Classification / Decision and Regression Trees}. We will use \ref{decision_trees} \nameref{decision_trees} and \ref{random_forest} \nameref{random_forest} to do classification on response variable \textbf{y}, i.e. whether the client subscribed a term deposit. \\
    \\
    The following steps will be performed to complete this section:
    \begin{enumerate}
        \item process data explanatory data analysis (EDA)
        \item data preparation for modeling
        \item visualization of random forest and decision tree
        \item cross validation and grid search
        \item comparison of the performance of random forest and decision tree
        \item conclusions
    \end{enumerate}
    
    \subsection{Process Data and Explanatory Data Analysis (EDA)}
    Import libraries
\begin{lstlisting}[language = Python]
import pandas as pd
import numpy as np
from datetime import datetime
import time
import gc
from IPython.display import display
import warnings
warnings.filterwarnings("ignore")

from sklearn.model_selection import train_test_split, GridSearchCV, StratifiedKFold
from sklearn.ensemble import RandomForestClassifier
from sklearn import tree
from sklearn.pipeline import Pipeline
from sklearn.metrics import roc_curve, roc_auc_score
import category_encoders as ce

import plotly.graph_objs as go
import matplotlib.pyplot as plt
\end{lstlisting}
\quad \\
    Loading data and a short explanatory data analysis
\begin{lstlisting}[language = Python]
# Process data
data = pd.read_csv('../../data/bank-additional-full.csv', sep=';')
display(data.sample(10))
display('There are {} observations with {} features'.format(data.shape[0], data.shape[1]))
\end{lstlisting}
\quad \\
    \begin{tabular}{lrlllllllcl}
        {} &  age &          job &   marital &          education & default & housing & loan &   contact & ... & y\\
        
        35666 &   58 &   management &   married &  university.degree &      no &      no &   no &  cellular & ... & no \\
        36531 &   48 &       admin. &   married &  university.degree &      no &     yes &   no &  cellular & ... & yes \\
        38676 &   73 &      retired &   married &  university.degree &      no &      no &  yes &  cellular & ... & yes \\
        15146 &   49 &   technician &  divorced &        high.school &      no &      no &   no &  cellular & ... & no \\
        33230 &   33 &  blue-collar &    single &           basic.6y &      no &     yes &   no &  cellular & ... & no \\
    \end{tabular} \\ \\
    \texttt{There are 41188 observations with 21 features}
    
    \subsection{Explore Categorical Variables in Dataset}
\begin{lstlisting}[language = Python]
# Function of plotting the categorical values disribution
def plot_bar(column):
    temp = pd.DataFrame()  #create a tremp dataframe
    temp['Not_deposit'] = data[data['y'] == 'no'][column].value_counts() # count the value when y = no
    temp['Deposit'] = data[data['y'] == 'yes'][column].value_counts() # count the value when y = yes

    temp = temp.apply(lambda x: x / x.sum(), axis = 1) * 100
    temp.sort_values("Deposit", inplace = True)
    temp.plot(
        kind = "barh", 
        stacked = True, 
        title = "Percentage Stacked Bar Graph for {}".format(column), 
        mark_right = True
    ).legend(loc = "center left", bbox_to_anchor = (1, 0.5))

    plt.savefig("../../plot/classification/{}.pdf".format(column), bbox_inches = "tight")
    plt.show();

plot_bar('job'), plot_bar('marital'), plot_bar('education'), plot_bar('contact'), plot_bar('loan'), plot_bar('housing')
\end{lstlisting}
    
    \begin{figure}%
        \centering
        \subfloat[job]{\includegraphics[scale=0.5]{plot/classification/job.pdf}}%
        \qquad
        \subfloat[marital]{\includegraphics[scale=0.5]{plot/classification/marital.pdf}}%
    
        \subfloat[education]{\includegraphics[scale=0.4]{plot/classification/education.pdf}}%
        \qquad
        \subfloat[contact]{\includegraphics[scale=0.4]{plot/classification/contact.pdf}}%
        
        \label{fig:example}%
    \end{figure}
    
    \subsection{Decision Trees} \label{decision_trees}
    Using the 20 explanatory variables above, it is desirable to fit a decision tree model to do prediction.
\begin{lstlisting}[language = Python]
import numpy as np
np.rand(5)
\end{lstlisting}
    
    \subsection{Random Forest} \label{random_forest}
    
    
    
    \newpage
    \section{Bibliography}
    \bibliographystyle{plain} % We choose the "plain" reference style
    \bibliography{refs} % Entries are in the refs.bib file

\end{document}
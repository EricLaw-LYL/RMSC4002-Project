\documentclass[11pt,a4paper]{article}
\input{myPreliminary}

\newcounter{magicrownumbers}
\newcommand\rownumber{\stepcounter{magicrownumbers}\arabic{magicrownumbers}}


%------------------------------------------------------------------------------



\begin{document}
    % Page 0 for names and table of contents
    \thispagestyle{empty}
    \pagenumbering{gobble} 
    \title{\textsc{RMSC 4002} -- Financial Data Analytics with Machine Learning \\ Group Project}
    \author{
        CHOI Sen Hei (SID: \texttt{1155109412}) \\
        IEONG Hei (SID: \texttt{1155104271}) \\
        LAM Wai Chui (SID: \texttt{1155152095}) \\
        LAU Chiu Tan (SID: \texttt{1155108960}) \\
        LAW Yiu Leung Eric (SID: \texttt{1155149315})
    }
    \date{\today}
    \maketitle
    
    \tableofcontents
    \newpage
    
    
    % Section 1
    \pagenumbering{arabic}
    \setcounter{page}{1}
    
    \section{Principal Component Analysis Factor Models and Recommender Systems}
    \subsection{Pricipal Component Analysis (PCA)}
    
    
    % Section 2
    \newpage
    \section{Classification / Decision and Regression Trees}
    In this section, we use Decision Trees and Random Forest methods to predict binary response variable.
    
    \subsection{Dataset}
    For the dataset, we use \href{https://archive.ics.uci.edu/ml/datasets/bank+marketing}{Bank Marketing Data Set} from  \href{https://archive.ics.uci.edu/ml/index.php}{UCI Machine Learning Repository}.
    The data is related with direct marketing campaigns of a Portuguese banking institution. The marketing campaigns were based on phone calls. Often, more than one contact to the same client was required, in order to access if the product (bank term deposit) would be ('yes') or not ('no') subscribed \cite{MORO201422}. There are total 21 variables (including y, the response variable), 41188 valid records.
    
    \subsection{Feature Description}
    There are 20 explanatory variables:
    \begin{table}[h]
        \begin{minipage}{.5\linewidth}
            \centering
            \begin{tabular}{r c c}
                 & Feature & Type \\
                \hline \hline
                \rownumber & age & numeric \\
                \rownumber & job & categorical \\
                \rownumber & marital & categorical \\
                \rownumber & education & categorical \\
                \rownumber & default & categorical \\
                \rownumber & housing & categorical \\
                \rownumber & loan & categorical \\
            \end{tabular}
            \caption{Bank client data}\label{tab:bank.client}
        \end{minipage}%
        \begin{minipage}{.5\linewidth}
            \centering
            \begin{tabular}{r c c}
                 & Feature & Type \\
                \hline \hline
                \rownumber & contact & categorical \\
                \rownumber & month & categorical \\
                \rownumber & day\_of\_week & categorical \\
                \rownumber & duration & numerical \\
            \end{tabular}
            \caption{Related with the last contact of the current campaign}\label{tab:last.contact}
        \end{minipage}
    \end{table}
    
    \begin{table}[h]
        \begin{minipage}{.5\linewidth}
            \centering
            \begin{tabular}{r c c}
                 & Feature & Type \\
                \hline \hline
                \rownumber & campaign & numeric \\
                \rownumber & pdays & numerical \\
                \rownumber & previous & numerical \\
                \rownumber & poutcome & categorical \\
            \end{tabular}
            \caption{Other attributes}\label{tab:other}
        \end{minipage}%
        \begin{minipage}{.5\linewidth}
            \centering
            \begin{tabular}{r c c}
                 & Feature & Type \\
                \hline \hline
                \rownumber & emp.var.rate & numerical \\
                \rownumber & cons.price.idx & numerical \\
                \rownumber & cons.conf.idx & numerical \\
                \rownumber & euribor3m & numerical \\
                \rownumber & nr.employed & numerical \\
            \end{tabular}
            \caption{social and economic context attributes}\label{tab:soc.econ}
        \end{minipage}
    \end{table}
    
    \subsection{Decision Trees}
    \begin{lstlisting}[language = Python]
import numpy as np
np.rand(5)
    \end{lstlisting}
    
    
    \newpage
    \section{Bibliography}
    \bibliographystyle{plain} % We choose the "plain" reference style
    \bibliography{refs} % Entries are in the refs.bib file

\end{document}
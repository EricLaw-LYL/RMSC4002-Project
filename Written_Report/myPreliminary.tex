%%%----------------------------------------------------------------------------%%%
%%%----------------------------------------------------------------------------%%%
%%% 
%%% ### Packages 
%%%
%%%----------------------------------------------------------------------------%%%
%%%----------------------------------------------------------------------------%%%
\usepackage{amsmath, amsthm, amssymb, nccbbb, bm, dsfont, pifont, fontawesome, graphicx, varioref, enumitem, mathtools, listings, xcolor, cite, fancyhdr, lastpage}
\RequirePackage[round,authoryear]{natbib}
\RequirePackage[colorlinks,citecolor=blue,urlcolor=blue]{hyperref}
\usepackage[cal=boondox]{mathalfa}
%\usepackage[numbers]{natbib}
\setlength{\topmargin}{-0.5in}
\setlength{\textheight}{10in}
\setlength{\oddsidemargin}{-.6in}
\setlength{\textwidth}{7.5in}
\hypersetup{colorlinks=true, linkcolor=blue, citecolor=red, urlcolor=blue}

%%%----------------------------------------------------------------------------%%%
%%%----------------------------------------------------------------------------%%%
%%% 
%%% ### Code 
%%%
%%%----------------------------------------------------------------------------%%%
%%%----------------------------------------------------------------------------%%%
\usepackage{listings}
\lstset{escapeinside=| |}
\usepackage[usenames,dvipsnames]{color}  
\definecolor{mygray}{rgb}{0.99,0.99,0.99}
\definecolor{myblue}{rgb}{0.0, 0.23, 0.63}
\definecolor{myred}{rgb}{0.75, 0.0, 0.0}
\definecolor{mygreen}{rgb}{0.4, 0.69, 0.2}  
\lstnewenvironment{R}{\lstset{ 
  language=R,
  basicstyle=\footnotesize\ttfamily, 
  numbers=left,
  numberstyle=\tiny\color{black},
  stepnumber=1,
  numbersep=5pt,
  backgroundcolor=\color{mygray},
  showspaces=false, 
  showstringspaces=false,
  showtabs=false, 
  frame=single,  
  rulecolor=\color{black},
  tabsize=4,
  captionpos=b,
  breaklines=true,
  breakatwhitespace=false,
  keywordstyle=\ttfamily\bfseries\color{myblue},
  commentstyle=\ttfamily\bfseries\color{myred},
  stringstyle=\ttfamily\bfseries\color{mygreen}
} 
}{}

%New colors defined below
\definecolor{codegreen}{rgb}{0,0.6,0}
\definecolor{codegray}{rgb}{0.5,0.5,0.5}
\definecolor{codepurple}{rgb}{0.58,0,0.82}
\definecolor{backcolour}{rgb}{0.95,0.95,0.92}

%Code listing style named "mystyle"
\lstdefinestyle{mystyle}{
  backgroundcolor=\color{backcolour}, commentstyle=\color{codegreen},
  keywordstyle=\color{magenta},
  numberstyle=\tiny\color{codegray},
  stringstyle=\color{codepurple},
  basicstyle=\ttfamily\footnotesize,
  breakatwhitespace=false,         
  breaklines=true,                 
  captionpos=b,                    
  keepspaces=true,                 
  numbers=left,                    
  numbersep=5pt,                  
  showspaces=false,                
  showstringspaces=false,
  showtabs=false,                  
  tabsize=2
}

%"mystyle" code listing set
\lstset{style=mystyle}

%%%----------------------------------------------------------------------------%%%
%%%----------------------------------------------------------------------------%%%
%%% 
%%% ### Box
%%%
%%%----------------------------------------------------------------------------%%%
%%%----------------------------------------------------------------------------%%%
\usepackage{tcolorbox}
\tcbuselibrary{breakable}
\newtcolorbox{mybox}{colback=yellow!5!white, colframe=gray!60!black, breakable}
\newtcolorbox{mybox0}{colback=white, colframe=gray!60!black, breakable}
	

%%%----------------------------------------------------------------------------%%%
%%%----------------------------------------------------------------------------%%%
%%% 
%%% ### Theorem style structures 
%%%
%%%----------------------------------------------------------------------------%%%
%%%----------------------------------------------------------------------------%%%
\numberwithin{equation}{section}
\theoremstyle{plain}
\newtheorem{theorem}{Theorem}[section]
\newtheorem{lemma}[theorem]{Lemma}
\newtheorem{corollary}[theorem]{Corollary}
\newtheorem{proposition}[theorem]{Proposition}
\newtheorem{condition}{Condition}[section]
\newtheorem{definition}{Definition}[section]
\theoremstyle{definition}
\newtheorem{example}{Example}[section]
\newtheorem{exercise}{Exercise}[section]
\newtheorem{remark}{Remark}[section]
\newtheorem{remark0}{Remark}
\newtheorem{question}{Question}


%%%----------------------------------------------------------------------------%%%
%%%----------------------------------------------------------------------------%%%
%%% 
%%% ### Operators 
%%%
%%%----------------------------------------------------------------------------%%%
%%%----------------------------------------------------------------------------%%%
\newcommand{\pr}{\mathsf{P}} 
\newcommand{\E}{\mathsf{E}} 
\newcommand{\median}{\mathop{\mathsf{median}}}
\newcommand{\Cov}{{\mathsf{Cov}}} 
\newcommand{\Corr}{{\mathsf{Corr}}} 
\newcommand{\Var}{{\mathsf{Var}}}
\newcommand{\SD}{{\mathsf{SD}}}
\newcommand{\CV}{{\mathsf{CV}}}
\newcommand{\Bias}{{\mathsf{Bias}}}
\newcommand{\AMSE}{\operatorname{\mathsf{AMSE}}}
\newcommand{\MSE}{\operatorname{\mathsf{MSE}}}
\newcommand{\ARE}{\mathsf{ARE}}
\newcommand{\AV}{\mathsf{AV}}
\newcommand{\CRLB}{{\mathsf{CRLB}}}

\newcommand{\pCorr}{\text{P}}
\newcommand{\sCorr}{\text{S}}
\newcommand{\kCorr}{\text{K}}
\newcommand{\bdCorr}{\text{BD}}
\newcommand{\cCorr}{\text{C}}


\newcommand{\inD}{    \overset{ \textnormal{d}   }{\rightarrow} }
\newcommand{\inAS}{   \overset{ \textnormal{a.s.}   }{\rightarrow} }
\newcommand{\inP}{    \overset{ \textnormal{pr}    }{\rightarrow} }
\newcommand{\inLp}{   \overset{ \mathcal{L}^p }{\rightarrow} }
\newcommand{\inMSE}{  \overset{ \textnormal{qm} }{\rightarrow} }
\newcommand{\inQM}{   \overset{ \textnormal{qm} }{\rightarrow} }
\newcommand{\indep}{\protect\mathpalette{\protect\independenT}{\perp}}
\def\independenT#1#2{\mathrel{\rlap{$#1#2$}\mkern4mu{#1#2}}}
\newcommand{\iid}{\textsc{iid}} 
\newcommand{\simIID}{   \overset{ \iid   }{\sim} }
\newcommand{\simIND}{   \overset{ {\indep}   }{\sim} }


\newcommand{\Bern}{\textnormal{Bern}} 
\newcommand{\Unif}{\textnormal{Unif}} 
\newcommand{\Normal}{\textnormal{N}} 
\newcommand{\logNormal}{\textnormal{LN}} 
\newcommand{\Bin}{\textnormal{Bin}} 
\newcommand{\NB}{\textnormal{NB}} 
\newcommand{\HG}{\textnormal{HG}} 
\newcommand{\Geom}{\textnormal{Geom}} 
\newcommand{\Beta }{\textnormal{Beta}} 
\newcommand{\BetaBin}{\textnormal{Beta-Bin}}
\newcommand{\Ga}{\textnormal{Ga}} 
\newcommand{\Exp}{\textnormal{Exp}} 
\newcommand{\Expo}{\textnormal{Expo}} 
\newcommand{\Po}{\textnormal{Po}} 
\newcommand{\Multi}{\textnormal{Multi}} 
\newcommand{\student}{\textnormal{t}} 
\newcommand{\Cauchy}{\textnormal{Cauchy}} 
\newcommand{\Pareto}{\textnormal{Pareto}} 
\newcommand{\Laplace}{\textnormal{Laplace}} 
\newcommand{\Logistic}{\textnormal{Logistic}} 
\newcommand{\Dir}{\textnormal{Dir}} 
\newcommand{\DP}{\textnormal{DP}} 
\newcommand{\Inv}{\textnormal{Inv-}} 
\newcommand{\F}{\textnormal{F}} 
\newcommand{\sign}{\textnormal{sign}}
\newcommand{\rank}{\textnormal{rank}}


\newcommand{\RV}{\textsc{rv}}
\newcommand{\cdf}{\textsc{cdf}} 
\newcommand{\cgf}{\textsc{cgf}} 
\newcommand{\pdf}{\textsc{pdf}} 
\newcommand{\pmf}{\textsc{pmf}} 
\newcommand{\chf}{\textsc{chf}} 
\newcommand{\mgf}{\textsc{mgf}}
\newcommand{\EF}{\textsc{EF}}
\newcommand{\NEF}{\textsc{NEF}}
\newcommand{\MLE}{\textsc{mle}}
\newcommand{\MAP}{\textsc{MAP}}
\newcommand{\Med}{\textsc{Med}}
\newcommand{\MME}{\textsc{mme}}
\newcommand{\QME}{\textsc{qme}}
\newcommand{\UMVUE}{\textsc{umvue}}
\newcommand{\MPT}{\textsc{MPT}}
\newcommand{\UMPT}{\textsc{UMPT}}
\newcommand{\LRT}{\textsc{LRT}}


\newcommand{\diag}{\mathop{\mathrm{diag}}}
\newcommand{\tr}{\mathop{\mathrm{tr}}}
\newcommand{\T}{\mathop{\mathrm{T}}}
\DeclareMathOperator*{\argmin}{arg\,min}
\DeclareMathOperator*{\argmax}{arg\,max}
\DeclareMathOperator{\sgn}{sgn}
\DeclareMathOperator{\logit}{logit}
\DeclareMathOperator{\expit}{expit}
\newcommand{\dd}{\textnormal{d}}


\newcommand{\lva}{{\color{myred}\ding{73}\ding{73}\ding{73}}}
\newcommand{\lvb}{{\color{myred}\ding{72}\ding{73}\ding{73}}}
\newcommand{\lvc}{{\color{myred}\ding{72}\ding{72}\ding{73}}}
\newcommand{\lvd}{{\color{myred}\ding{72}\ding{72}\ding{72}}}
\newcommand{\optional}{\noindent{\color{myblue}\faScissors}}
\newcommand{\Solution}{\noindent{\color{myblue}{{\textsc{Solution}}:~$\Big.$}}}
\newcommand{\take}{\noindent{\color{myblue}\faPaperPlaneO~\underline{\bf Takeaway}:~$\Big.$}}

% widecheck 
\DeclareFontFamily{U}{mathx}{\hyphenchar\font45}
\DeclareFontShape{U}{mathx}{m}{n}{
      <5> <6> <7> <8> <9> <10>
      <10.95> <12> <14.4> <17.28> <20.74> <24.88>
      mathx10
      }{}
\DeclareSymbolFont{mathx}{U}{mathx}{m}{n}
\DeclareFontSubstitution{U}{mathx}{m}{n}
\DeclareMathAccent{\widecheck}{0}{mathx}{"71}
\DeclareMathAccent{\wideparen}{0}{mathx}{"75}

\def\cs#1{\texttt{\char`\\#1}}

% Customize header and footer
\pagestyle{fancy}
\fancyhf{}
\fancyhead[L]{2021-22 Semester 1}
\fancyhead[R]{RMSC4002 Financial Data Analytics with Machine Learning}
\fancyfoot[C]{Page \thepage\ of \pageref*{LastPage}}